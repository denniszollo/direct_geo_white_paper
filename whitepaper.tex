\documentclass{article}

\usepackage{swiftnav}


\usepackage{draftwatermark}
\SetWatermarkLightness{0.9}
\SetWatermarkText{Preliminary}
\SetWatermarkScale{1}

% Suppress numbers from section headings (but preserve PDF TOC).
\makeatletter
\renewcommand\@seccntformat[1]{}
\makeatother

\newenvironment{mpar}{\par\noindent\minipage{\linewidth}}{\endminipage\par}
%\setlength{\skip\mpfootins}{2cm}
\renewcommand{\thempfootnote}{(\arabic{mpfootnote})}

% ---------------------------------------------------------------------------

\version{2.3.1}
\title{Piksi for UAV Aerial surveying}
\mysubtitle{RTK direct georeferencing with Swift Navigation's Piksi GPS receiver}
\author{Dennis Zollo, Rai Gohalwar}
\date{\today}

\begin{document}

\maketitle

\thispagestyle{firstpage}

\section{Abstract}
\label{sec:abstract}
This whitepaper presents results of using the Piksi RTK GPS sensor for georeferencing of aerial images taken aboard a small UAV.  
It presents an integration strategy of the Piksi sensor, as well as the results of a survey flight as analyzed by the PIX4D commercial software. 
Different data collection and post-processing techniques are presented with an attempt to show potential value and drawbacks of using an RTK position sensor for this application.
\tableofcontents
\newpage
\section{Overview}
\label{sec:Overview}
This is the overview
\section{Equipment and Setup}
\label{sec:equipment}
here we describe the equipment
\thispagestyle{lastpage}
\end{document}
